\section{Introduction}
\label{sec:intro}

Web applications have proliferated and found wide use because it is
especially easy for developers to reuse components of existing web
applications to create new ones. While component reuse in the
venerable desktop software model typically takes the form of
libraries, the reusable components in web applications aren't limited
to just JavaScript library code---they further include
network-accessible content and services (e.g., map data and
storage). Such reuse is ubiquitous. jQuery, a widely used JavaScript
library, is included in and used by 68\% of the Quantcast top-10,000
web sites, and over 52\% of the Quantcast top-million web
sites~\cite{quantcast}.
\todo{ds}{update numbers}

The resulting model is one in which web application developers cobble
together multiple JavaScript libraries, web-based content, and
web-based services written and operated by various parties (some of
which may in turn incorporate libraries, content, and services from
still other sources) and add their own JavaScript code that integrates
these components and builds the required application-specific
functionality atop them. Such web applications are today commonly
referred to as {\em mashups.} A web application's behavior may be
enhanced and/or customized further by {\em extensions} the
user has installed into the browser, which in Firefox and Chrome take
the form of JavaScript code written by other parties
as well. Unfortunately, some of the many contributors to the tangle of
JavaScript comprising a web application may not have the user's best
interest at heart. Given that today's web applications manipulate such
sensitive data as email, bank statements, health records, and users'
passwords, there is great potential for violations of users' privacy
contributed by JavaScript miscreants.

Out of this state of affairs fall two goals for web
applications. First, {\em flexibility:} developers should be able to
flexibly compose services, content, and JavaScript code from multiple
parties into rich, featureful web applications. And second, {\em
  privacy:} there should be strong guarantees that a user's privacy
cannot be violated when a web application processes her sensitive
data. These goals are hardly new. Wang {\em et al.} articulated
similar ones, and proposed new browser primitives to improve isolation
within mashups: sandboxes that allow pages not to share their data
with JavaScript code they include, and discretionary access control
(DAC) for inter-frame communication~\cite{mashupos}. Indeed, today's
browsers incorporate similar mechanisms in the guises of HTML5's
seamless iframes, iframe sandbox, and postMessage API~\cite{html5}.
And the well known Same-Origin Policy (SOP, detailed in
Section~\ref{sec:background}) prevents JavaScript hosted by one principal from
reading content hosted by another.

Unfortunately, in the status quo web browser security architecture,
efforts to advance one of these two goals still often hamper achieving
the other. Application developers (and users) are thus often left with
only unpalatable choices (though in practice, the desire for
functionality often dominates the concern for privacy, to users'
peril). The browser may enhance privacy by applying discretionary
access control when an untrusted script wishes to read
potentially sensitive data. The various isolation mechanisms a modern
browser applies (e.g., the SOP and Content Security Policies (CSP),
both explored in Section~\ref{sec:background}) are flavors of DAC. But
the brittle ``all-or-nothing'' character of DAC thwarts the building
of many useful mashups. Broadly speaking, DAC can either deny or grant
untrusted JavaScript access to the sensitive data. In the former case,
that untrusted JavaScript might {\em need} the sensitive data to
implement the functionality of the application. In the latter case,
DAC exercises no control over what the untrusted code does with the
sensitive data.

Modern browsers also incorporate mechanisms that explicitly support
the prioritization of flexibility over privacy. For example, an
application developer may choose to include JavaScript from another
origin with a {\tt <script>} tag, which effectively bypasses the SOP,
interpolating the untrusted script into the enclosing page and
granting the script unfettered access to the enclosing page's origin's
content. Or a web server's operator may use Cross-Origin Resource
Sharing (CORS, also treated in Section~\ref{sec:background}) to
explicitly enumerate other origins permitted to read content from that
server (\Red{bypassing the SOP for those other origins}).
This approach still limits flexibility, though, in that it requires a
server's operator to know {\em a priori} which mashups are hosted by
which other origins, and worse, potentially allows a service provider
to limit users' choice of applications that integrate content from the
provider's site.\footnote{The user has the further unpalatable
  choice of providing her login credentials for the service
  provider's site to the web application---a potential
  privacy and integrity disaster for, say, one's online banking site,
  yet the precise model that services like {\tt mint.com} use for lack
  of a better alternative.} This, of course, also potentially sacrifices
privacy, because the other origins are typically operated by other
principals who may in turn leak any sensitive information they
receive.

To achieve both flexibility and privacy, what is needed is a means to
allow untrusted code to compute over sensitive data and display
results to  the user, while {\em confining} the untrusted code so that
it may not exfiltrate the sensitive data (in the web browser,
typically by writing the sensitive data to an untrusted remote
origin). To offer the greatest flexibility to web application
developers, the means of confinement should further be {\em first
  class,} in that any container (e.g., a frame) should be able
to prevent untrusted code incorporated in a sub-container from
exfiltrating sensitive data passed into the sub-container, with
arbitrarily deep nesting, e.g., so that a page may include a
library it does not trust, and that library may in turn include
another library {\em it} does not trust. And finally, also in the
interest of flexibility, the means of confinement should be {\em
  symmetric,} in that it should allow each principal on either side of
a trust boundary to pass sensitive data to the other, while preventing
the other principal from exfiltrating it.  
%
This property, for instance, ensures that the parent of a
sub-container cannot arbitrarily exfiltrate data it receives from the
sub-container.
%
No existing combination of browser security mechanisms offers these
properties.

In this paper, we present \sys{} (the Secure Web API), a simple,
label-based mandatory access control (MAC) system for web browsers
that delivers both flexibility for web application developers and
robust privacy for users. \sys{} allows untrusted JavaScript to
process sensitive data, and confines the untrusted JavaScript in
first-class, symmetric fashion.  To illustrate the value of these
properties, we present five case-study applications---a password
strength-checker, a password manager, a banking application that uses
the (untrusted) jQuery library, a ``phone numbers-to-links'' browser
extension, and a third-party mashup, none of which can be implemented
in a way that ensures privacy in today's web browsers.  We then build
them atop \sys. Our contributions include:

\begin{CompactItemize}
\item We characterize five case-study web applications
  (Section~\ref{sec:goals}) for which today's browser security
  architecture cannot provide strong privacy.
\item We describe the design of the \sys{} label-based MAC system
  for web browsers (Section~\ref{sec:system}), which meets the
  requirements of the five case-study web applications.
\item We describe designs of the five case-study web applications atop
  \sys.
\item We describe the implementation of \sys{}
  (Section~\ref{sec:implementation}) and evaluate the performance
  overhead in the Firefox and Chromium open-source browsers; our
  evaluation (Section~\ref{sec:eval}) illustrates that \sys{} incurs
  minimal performance overhead over the respective baseline browsers.
\end{CompactItemize}

% Local Variables:
% TeX-master: "main.ltx"
% TeX-command-default: "Make"
% End:
