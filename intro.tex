\section{Introduction}
\label{sec:intro}

Web applications have proliferated and found wide use because it is
especially easy for developers to reuse components of existing
applications to create new ones. While component reuse in the
venerable desktop software model typically takes the form of
libraries, the reusable components in web applications aren't limited
to just JavaScript library code---they further include
network-accessible content and services ({\em e.g.,} map data and
storage). The resulting model is one in which web application
developers cobble together multiple JavaScript libraries, web-based
content, and web-based services written and operated by various
parties (some of which may in turn incorporate libraries, content, and
services from still other sources) and add their own JavaScript code
that integrates these components and builds the required
application-specific functionality atop them. Such web applications
are today commonly referred to as {\em mashups.} A web application's
behavior may be enhanced and/or customized further by browser {\em
  extensions} the user has installed into the browser, which in
Firefox and Chrome take the form of JavaScript code written by other
parties still. Unfortunately, some of the many contributors to the
tangle of JavaScript comprising a web application may not have the
user's best interest at heart. Given that today's web applications
manipulate such sensitive data as email, bank statements, health
records, and users' passwords, there is great potential for violations
of users' privacy by JavaScript contributed by miscreants.

Out of this state of affairs fall two goals for web
applications. First, {\em flexibility:} developers should be able to
flexibly compose services, content, and JavaScript code from multiple
parties into rich, featureful web applications. And second, {\em
  privacy:} there should be strong guarantees that a user's privacy
cannot be violated when a web application processes her sensitive
data. Unfortunately, in the status quo web browser security
architecture, efforts to advance one of these goals unfortunately
often hamper achieving the other.

% include JQuery use stats: over 68% of Quantcast top 10k sites, and
% over 52% of Quantcast top million sites. (cite
% http://trends.builtwith.com/javascript/jQuery)
