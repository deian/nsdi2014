\section{Related Work}
\label{sec:related}

% BFlow
%The closest related work to our system is BFlow.
%
Like \sys{}, BFlow~\cite{Yip:2009:PBS} allows web sites to enforce confinement policies
stricter than the SOP\@.
%
BFlow can, for example, confine third-party code by placing
it in a tainted \emph{protection zone}---a group
of iframes that share a common label.
%
However, unlike \sys{}, BFlow cannot mediate between mutually
distrustful principals: e.g., one cannot directly implement the
encrypted document editor with BFlow.
%
This is because BFlow does not provides symmetric confinement---a
sub-frame cannot impose any restrictions on its parent---and its
labels do not support conjunctions of multiple origins (e.g., BFlow
cannot encode a policy that states that sensitive data may derive from
data from two web sites).\footnote{ The mutually distrusting scenario
  is further complicated by BFlow's reliance on top-level
  \emph{trusted zones} that can always \emph{completely} declassify
  the data in their corresponding page.}
%
For the same reasons, BFlow cannot support applications that require
security policies more flexible than the SOP, such as our third-party
mashup example.
%
These differences reflect different goals for the two systems: BFlow's
authors set out to confine untrusted third-party scripts, but never to
support applications that incorporate code from mutually distrusting
parties.
%
Like BFlow, \sys{} accepts labels that arrive in HTTP headers and is thus
amenable to confining one user's code from another's on the same
website. However, we have yet to develop a server that implements this
functionality; developers can use BFlow's server or systems such as
Hails~\cite{giffin:2012:hails} or Aeolus~\cite{cheng:aeolus} to
ensure that labels are propagated appropriately in an end-to-end
fashion.
%
%%I think the story might be even better for us: I don't think they
%%can build password checker where the latter is an iframe hosted on
%%other domain.
%%
%%Waiting on email

%% BFlow~\cite{Yip:2009:PBS} is a confinement system for web browsers. It tracks
%% flows of information at the granularity of \emph{secure zones}, i.e.,
%% compartments composed of one or several iframes.  {\sys} does not require such
%% abstraction, it relies on the compartment notions already provided by browsers,
%% i.e., iframes and workers. Similar to our approach, BFlow uses \js|postMessage|
%% for communication across secure zones. BFlow does not allow JavaScript code in
%% different secure zones to write to shared DOM variables and cookies regardless
%% their security labels.  While \sys~ supports more expressive labels, potentially
%% including different domains, BFlow can propagates labeled data to the server
%% side. It is stated as future work to achieve a similar feature.


% DCS 
More recently, Akhawe \emph{et al.}~proposed the data-confined
sandbox (DCS) system~\cite{Akhawe2013}, which provides pages with the
ability to intercept and monitor the network, storage, and cross-origin
channels of certain iframes.
%
Such monitoring can in principle confine code such as the
password-strength checker.
%
However, DCS's confinement is fundamentally limited to \verb|data:|
URI iframes, and, as a result, DCS cannot confine a service provided
in an iframe (the common case, both to isolate the enclosing page from
the script and because the checker's author may not want to share
source code\tocite{postman}).
%
Like BFlow, DCS does not offer symmetrical confinement---an iframe
cannot impose restrictions on its parent.
%
DCS's support for only asymmetric confinement falls out of the work's
narrower goal of confining untrusted libraries.
%
Unlike \sys{}, DCS is not limited to using labels for policy
specification; instead, it allows a page to implement policies using
arbitrary JavaScript.
%
However, to ensure that the monitor cannot reveal more information
about an iframe's content than allowed by the SOP, a page can only
impose a limited range of policies on an iframe (e.g., DCS cannot
impose restrictions on redirects).
%
\sys{} does not have this limitation---the iframe itself imposes
the policy by voluntarily raising its label.
%
\sys{}'s declarative policies further let developers safely relax the
SOP and build applications such as third-party mashups.
%

% \Red{What is the limitation} However, we recognize this as a limitation of \sys{} and leave the
% investigation of expressiveness labels to future work.

%% Recently, and different from other sandboxes approaches, the
%% data-confined sandbox system~\cite{Akhawe2013} (DCS) restricts
%% propagation of information by using \js|iframes| and mediating
%% cross-domain operations (e.g. access to local storage, fragment-IDs,
%% network communication, etc.).  Every sandbox communicates by
%% \js!postMessage! only with its designated parent. The parent defines
%% the confinement policy and plays the role of man-in-the-middle when a
%% sandbox wish to communicate with another one. While having similar
%% goals, DCS is more restrictive than \sys. DCS disallows dynamic code
%% evaluation inside sandboxes and presents a not egalitarian design,
%% i.e., a sandbox (\js|iframe|) cannot host another sandbox. Clearly,
%% the DCS client-side TCB is smaller than ours.



% JSFlow 
JSFlow~\cite{JSFlow} is a fine-grained IFC system for JavaScript
that is implemented in JavaScript.
%
Because JSFlow can associate labels with individual objects, a
developer can use JSFlow to confine untrusted libraries that are
closely coupled with trusted code on a page (as is jQuery).
%
\sys{}, on the other hand, is coarse-grained and, like BFlow, generally
requires that an application be compartmentalized (e.g., by placing the
page's code that needs to communicate with the network in a lightweight
DOM worker).
%
Despite these benefits, JSFlow's fine-grained approach
also incurs costs.
%
First, JSFlow cannot confine style sheets that execute code
or support applications that rely on policies more flexible than SOP,
such as third-party mashups that read cross-origin data (e.g., using
XHR).
%
%%Second, developers need to understand a new semantics for the core
%%JavaScript language~\cite{Hedin:2012}, in contrast to a new DOM-leve
%%API.
%
Second, as an IFC-supporting JavaScript interpreter written in
JavaScript,  JSFlow must model all library APIs, including
the JavaScript built-in APIs.
% That said, the coverage of the DOM demonstrated is commendable.
%
Executing pages' JavaScript code in an interpreter written in
JavaScript incurs a significant execution slowdown (by two orders of magnitude).
And the requirement that JSFlow include a model of all
library APIs restricts the APIs available to developers.
%
JSFlow is largely complementary to \sys{}; we believe there may be
applications where it would make sense for some \sys{} compartments to
enforce fine-grained IFC internally with JSFlow.

ConDOM~\cite{ConDOM} is another fine-grained IFC system for the
browser. Like JSFlow, ConDOM extends objects with labels that are
propagated across the JavaScript built-in API, and consequently shares
JSFlow's execution overhead.
% and  suffer from similar shortcomings as 
% the ones described for JSFlow. 


%% Hedin and Sabelfeld~\cite{Hedin:2012} formally describe a sound language-based
%% confinement mechanisms for a subset of JavaScript and their ideas are being
%% currently applied to JSFlow~\cite{JSFlow}. %, a modified JavaScript interpreter.
%% JSFlow is designed to confine legacy code, and because of that, it does an
%% impressive effort to obtain fine-grain labeling of data. This is achieved at the
%% price of proposing detailed models for capturing big parts of the browser
%% semantics (e.g. interaction with the DOM), which drastically impact on
%% performance (sometimes more than 100\%!).  We provide a more coarse-grain
%% approach: \emph{\sys{} does not care for most of the JavaScript and DOM API
%%   internals}. Instead, \sys~ uses the browsing contexts and simply
%% mediates among them to preserve security.


% With our
FlowFox~\cite{DeGroef:2012} is an IFC system that uses secure-multi
execution (SME)~\cite{Devriese:2010} to execute a program multiple
times, once per label; SME ensures that no leaks from a sensitive
context can leak into a less-senstive context by construction.
%
Like JSFlow, FlowFox can confine programs
with fine-grained labeling of data.
%
Unlike JSFlow and \sys{}, FlowFox's SME approach is not
amenable to scenarios where declassification plays
a key role (e.g., the encrypted editor, password manager, or
third-party mashup). 
%
Although SME has been recently extended to consider
declassification~\cite{Rafnsson:2013}, it is unclear how this
theoretical result translates into FlowFox, and more importantly, how
it affects its applicability.
%
Applications of FlowFox to mitigation of history sniffing and behavior
tracking are made possible by its labeling of user interactions and
metadata (history, screen size, etc.).
%
These attacks are outside the present threat model for \sys{}.  The
degree of protection \sys{} provides for such data is limited to the
policies imposed by developers before sharing the data.

%
%% Authors show how FlowFox can enforce
%% non-interference like policies for popular web cites. While \sys{} does not
%% target legacy code, it can enforce a wider-range of policies
%% (e.g. declassification). Different from \sys{}, FlowFox requires a total
%% ordering of the security lattice, an uncommon assumption in an scenario with
%% mutual distrust (as the web). While recent results show how to lift this
%% requirement in reactive systems (as the browser)~\cite{ZanariniJR13}, FlowFox
%% has not yet incorporate them into its design.

% ConDOM 
%\todo{ar}{revise as the above}
%ConDOM~\cite{ConDOM} implements fine-grained label tracking system
%that spans both the JavaScript engine and DOM\@.
%%
%To handle implicit flows, authors use a
%control flow stack (neglecting exceptions) and inject labels at the HTML parser
%for dynamically generated code. In contrast, \sys{} handles any type of
%branches or dynamic code by labeling the browsing context itself. \todo{ezy}{this doesn't seem very convincing\ldots}


%Sandboxing/language subsets
There is much work on sandboxing and developing subsets of JavaScript (e.g.,
Caja~\cite{GoogleCaja}, BrowserShield~\cite{Reis:2007},
WebJail~\cite{VanAcker:2011}, TreeHouse~\cite{Ingram:2012},
JSand~\cite{Agten:2012:JCC}, SafeScript~\cite{SafeScript}, Defensive
JavaScript~\cite{djs}, and recently~\cite{Howell:2013}). 
%
While our design has been inspired by some of these systems (e.g.,
TreeHouse), the usual goals of these systems are to mediate
security-critical operations, restrict access to the DOM, and restrict
communication APIs\@.
%
In contrast to the mandatory nature of confinement, however, most restrictions
are imposed in a discretionary fashion and are thus not suitable to the
building a number of the applications we consider (in particular, the encrypted
editor).
%
Nevertheless, we believe that access control and language subsets are a crucial
complement to confinement when building robust secure applications.



%%% Google extensions security analysis 
%%Carlini \emph{et al.}~\cite{Carlini:2012} evaluate the security of the
%%Google Chrome extension platform.
%
\sys{}'s lightweight DOM workers are very similar to content
scripts~\cite{Carlini:2012}, although provided as DOM objects to
website developers.
%
However, as opposed to real extension systems, we do not not consider
privileged APIs, nor do we intend to extend the API made available to
workers in such a way.
%
Instead, we believe that extension systems can stand to benefit from
confining content scripts, if only to provide a means for further
reducing the trust placed on existing extensions.
%
Indeed, since Firefox content scripts rely on the same core mechanisms
used by \sys{}, we expect porting the Add-on SDK to use confinement to
be require only modest effort.
 

% Local Variables:
% TeX-master: "main.ltx"
% TeX-command-master: "make"
% tex-dvi-view-command: "gmake preview;:"
% End:
