\section{Related Work}
\label{sec:related}



% Javascript Sandboxing 
Focusing on  different security policies, 
several language-based approaches has been divised to sandbox Javascript.  These
approaches usually work in a similar manner: they mediate security critical
operations and some of the browser API (e.g. the DOM).  
%
JSand wraps setters (\js|set|) and getters (\js|get|)  of Javascript
objects. Additionally, it propagates policies to newly created objects via  
a membrane pattern~\cite{XX}. Application-specific DOM-nodes are 
freely accessed withtin the sandbox. For global properties
(e.g. \js|window.document|), JSand simply wraps the \js|window| object in 
order to enforce a given policy.
% 
Similar to our work, TreeHouse uses workers to provide a fresh and isolated
execution context for Javascript code. Workers communicate by \js|postMessage|,
which guarantees that no references to outside objects can be passed into the
sandbox. TreeHouse provides virtualized DOM-nodes with the
restriction that workers cannot share them. 






Aiming to target legacy code, the main drawback of these approaches is two fold:
performance and completeness. While exposing significant performance degradation
in micro benchmarking (in some cases more than 400\%!), authors claim that the
impact on the user experience is acceptable.  On the completeness side,
JSand parses dynamically loaded scripts either asynchronously, potentially
changing the semantics of the application, or through a library, which might
interpret them differently from the browser. 
\Red{TODO:The problem with TreeHouse mentioned by Deian?}



%

%


% Local Variables:
% TeX-master: "main.ltx"
% TeX-command-master: "make"
% tex-dvi-view-command: "gmake preview;:"
% End:
