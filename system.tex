\section{The \sys{} confinement system}
\label{sec:system}

%%In this section, we describe the \sys{} system.
%%%
%%As mentioned previously, the basic unit of confinement in \sys{}---a
%%compartment---corresponds to a JavaScript execution context, e.g.\@ a
%%browser tab, an iframe on a page, or even a Web Worker~\cite{workers}.

\sys{} builds on the existing web security model by using compartments
as the basic unit of isolation and mediating communication between 
the compartments and external servers.
%
%
The two basic changes to the existing compartment model are as follows.
%
First, \sys{} associates each compartment with a MAC label.
%
The compartment label (Section~\ref{sec:labels}), indicates the
sensitivity of the data in the compartment; this label is used to
dictate where information can flow.\footnote{Conceptually, the notion
of labels extends to network end-points, i.e., web servers.}
%
%
%%In the case of browsing context, \emph{DOM label}, is further used to
%%indicate the sensitivity of the DOM.
%
Second, \sys{} makes the ambient authority of an origin explicit, with
privileges (Section~\ref{sec:privileges}).
 
\sys{} uses these labels and privileges to restrict the
kind of messages and requests which are permitted to flow to other
compartments and web servers, respectively.
%
However, since \sys{} re-purposes existing notions of compartments, we
must mediate existing communication mechanisms to ensure confinement.
%
(This is subtle when considering browsing contexts, which can use
different parts of the DOM API to exfiltrate data.)
%
The communication channels we interpose on include:
%
{\tt postMessage},
cross-compartment DOM objects,
content loading (via {\tt img} tags, CSS, etc.),
form submission,
XMLHttpRequest,
navigation, and
browser storage (cookies and local storage).
%% \begin{CompactItemize}
%%     \item {\tt postMessage},
%%     \item Cross-compartment DOM objects,
%%     \item Content loading (via {\tt img} tags, CSS, etc.),
%%     \item Form submission,
%%     \item XMLHttpRequest,
%%     \item Navigation, and
%%     \item Browser storage (cookies and local storage).
%% \end{CompactItemize}

However, this interposition maintains backwards-compatibility with
existing web policies.
%
Indeed, unless a compartment chooses to use \sys{} (e.g., by using our
API or communicating with a \sys{}-enabled compartment) the policy
imposed should be SOP.
%
Consequently, \sys{} policies are applied on top of the old
same-origin policy for existing browser APIs; this allows developers
to apply more restrictive policies than SOP\@.
%
However, to relax the same-origin policy, \sys{} also introduces
variants of these APIs---namely labeled XHR
(Section~\ref{sec:labeled-xhr}).\footnote{Relaxing the same-origin
  policy in a way that's secure and maintains backwards-compatibility
  is subtle, and we directly address this issue in
Section~\ref{sec:labeled-xhr}}

To provide developers with a means for privilege-separating
\js|<script>|'s, which always execute with the privilege of the page,
\sys{} also provides a new unit of confinement: a \emph{lightweight worker}
(LWorker)
%
LWorkers are compartments without a DOM, similar to existing Workers,
but, unlike normal Workers, execute in the same thread as their
parent.
%
When they are used to access DOMs (e.g., when confining jQuery) of
their parent (something that cannot be done with traditional Web
Workers) we call them ``DOM workers.''

%%The security policy of a container consists of three components:
%%
%%\begin{itemize}
%%    \item A current label (Section~\ref{sec:labels}), which indicates the
%%        sensitivity of the data in the container,
%%    \item A DOM label, which indicates the sensitivity of the DOM if the
%%        container is associated with one, and
%%    \item A set of privileges , which
%%        dicates what authority the compartment's code has to
%%        \emph{relax} policy requirements.
%%%   \item A clearance (Section~\ref{sec:clearance}), which imposes a
%%%       limit on the sensitivity of the data in the container.
%%\end{itemize}

\subsection{Labels}
\label{sec:labels}

The \emph{current compartment label} is the security policy which protects all data associated with the compartment or
web site, specifying what origins can read the data.
%
An additional \emph{DOM label} specifically protects the DOM associated
with a compartment, if it has one.
%
Individual objects can also be labeled: \sys{}
introduces opaque \emph{labeled blobs} for this purpose.
%
Labeled blobs are primarily used to label \js|postMessage| messages
(otherwise they would take the label of the compartment).
%
The particular labels that are used in \sys{} are \emph{DC labels},
presented in~\cite{stefan:2011:dclabels}.
%
We chose DC labels, much like Hails~\cite{giffin:2012:hails} and
Breeze~\cite{Breeze13}, because they have a simple
semantics while being as expressive as the labels used by other practical
confinement systems~\cite{GenLabels}.
%
%In this section, we briefly describe DC labels. %; an interested
%reader should see~\cite{stefan:2011:dclabels} for more details.

A privacy label (henceforth just label) is a boolean formula over
origins.%
%
\footnote{
  Our system also handles \emph{integrity}, but for simplicity we
  omit it from the main text, commenting on it in footnotes when
  appropriate.  Integrity is dual to secrecy, i.e., the logical
  implication is run the other direction.
}
%
When a compartment wishes to send a message to another compartment or
server, we check if the target label $l'$ logically implies the source
label $l$.
%
Intuitively, this requires that the security of the destination be as strong as original security policy, i.e., that the target label \emph{subsumes} the source label.\footnote{If the communication is bidirectional, e.g.\ mutation of a shared DOM, then both the target and source must imply each other.}
%
We also refer to the target label as \emph{higher} (as in ``higher security'') than the source label.
%
For example, to read the contents of a labeled blob, we must check that
the label of the compartment subsumes the label of the blob.
%
In the password checker, the labeled blob containing the password (2)
cannot be read by a publicly labeled worker (corresponding to the
logical formula ``true''), because ``true'' does not imply
``\https{fb.com}''.
%
However, once the worker raises its label to ``\https{fb.com}'', the
logical implication holds trivially.

%Reworded version: Intuitively, an entity is allowed to receive data that is at most as
%sensitive as its label, i.e., the entity's label always \js|subsumes|
%the label of the data it can receive, preserving the latter's privacy.
%   \Red{maybe something about how this makes it first
%   class: the fact that policies are preserved means that if a library has
%   sub-libraries, they will get the appropriate policy enforced}
%
%Why is a label represented as a boolean formula rather than simply
%as a set of origins?
%%
%As it turns out, both conjunctions (``and'')
%and disjunctions (``or'') are useful for specifying security policies.
%%
%Informally, a conjunction describes what origins contributed sensitive data
%to the compartment, while a disjunction describes what origins are allowed
%to read the data.
%%
%% EZY: Unfortunately, I can't use the password checker here because
%% the labels are too simple
%A compartment with the label ``\https{bank.ch} and \https{amazon.com}''
%may contain sensitive data from both sites, so it cannot send data to
%\https{bank.ch}, as doing so might leak private information from
%\https{amazon.com} (and vice versa).
%%
%Conversely, a compartment labeled ``\https{bank.ch} or \https{amazon.com}''
%could send data to either website; however, it could not receive any
%messages from either origin---it might have originally been public data
%whose sensitivity was raised so that no other site could read it.

How is the current label initially determined?
%
\sys{} extends the API of iframes and workers so that when they are
created, a particular current label can be assigned to it (which
must subsume the parent's label).  Otherwise, new containers implicitly
inherit the label of its parent.\footnote{Perceptive readers may now be wondering how
the password checker arranged to create a publicly labeled worker
when its label was \https{fb.com}.  In fact, doing so is not possible
to do so without privileges (Section~\ref{sec:privileges}).}
%
For web pages with no parent, we look for a {\tt Document-label} header which
specifies the initial label, defaulting to public (i.e., no extra
restrictions) when this header is not present.
%
The DOM label of a compartment is more stringent: it is set such that
only compartments with the privilege (Section~\ref{sec:privileges}) for the origin can write to
it.\footnote{This is done by setting the integrity label to the origin.}

After creation, a compartment can freely \emph{raise} its label, i.e.,
change its label to any label which subsumes it, in order to read more
sensitive data (from another compartment) but not vice versa.
%
This leads to a common interpretation of labels as ``taint;'' as
a compartment receives data from more sources, it accumulates more
origins on account of raising its label to read the received data.

\subsection{Privileges}
\label{sec:privileges}

% Description of privileges
A \emph{privilege} is an object, representing an origin, which confers
the authority to relax policy requirements with respect to this origin.
%
A compartment is associated with a set of privileges.
%
While privileges can be used in a variety of ways, their basic mechanism
is quite simple:  whenever a compartment with a privilege $p$ performs a
label check, instead of checking that $l'$ implies $l$, we check that
$l'$ and $p$ implies $l$~\cite{stefan:2011:dclabels}.

There are three primary use-cases for privileges:

\begin{itemize}
    \item A privilege can be used to \emph{declassify} private
        information without raising the current label.  For example, we
        could downgrade ``\https{amazon.com}'' labeled data to
        ``public''.

    \item A privilege can be used to lower the current label of
        a compartment (or a compartment that is being created).
        This is how privileges are used
        in the password checker: the untrusted worker was created
        with a current label of ``public'' by exercising the
        \https{fb.com} privilege in the main web site.  Conceptually, we
        are declassifying \emph{all} of the data associated with the
        compartment.

    \item A privilege must be exercised to edit the DOM
        of another compartment.\footnote{This is because the current
        integrity label of a compartment generally public, which
        without the privilege is insufficient to permit writes.}
\end{itemize}

In \sys{}, privileges are \emph{implicitly} exercised: if a compartment
has a privilege, it will always attempt to use it.
%
Importantly, however, code can retrieve and drop its privileges; this
allows programmers to wrap functions, such as \js|postMessage|, to
enforce a model wherein privileges must be explicitly exercised.
%
In addition, labeled blobs can be useful for avoiding privilege
exercise, by requiring that the \emph{target} compartment (which may
not have the privilege) do an additional label check.
%
In the password checker, the main
page must do precisely this, as it has the \https{fb.com} privilege.
%
Without labeling the password, it could accidentally leak the password
to a publicly labeled worker.

% Initial privileges and privilege creation
By default, a browsing context has the privilege corresponding to the
page origin; this context can drop the privilege if it is not
needed or spawn other containers sans privilege to run untrusted code.
%
While privileges for existing origins cannot be created, a privilege can
be created for a fresh origin, i.e., an origin which does not correspond
any actual web site, but which is guaranteed to be unique.
%
A fresh origin is often used to run code under full isolation, since any
compartment labeled with the fresh origin cannot communicate with any
other compartment without exercising the corresponding privilege.
%
Importantly, any code can allocate a fresh origin and its corresponding privilege: this
makes isolation an egalitarian security mechanism: an untrusted library
can use this mechanism to isolate a sub-library that it itself
may not trust~\cite{Zeldovich:2006}.

\subsection{Labeled XMLHttpRequest}
\label{sec:labeled-xhr}

In the previous sections, we described the general design principles
behind the security policies on compartments.
%
In this section, we concretely describe the implication of these
policies on one particular API function: XMLHttpRequest (XHR).
%
XHR is an interesting because: (1) the added \sys{} policy checks are
representative of those of other web browser APIs, (2) the API
interacts with servers, so it is important to explain how HTTP headers are
handled, and (3) the variant of this API called \emph{labeled XHR} is
the only component of \sys{} which relaxes the same-origin policy.

A traditional XHR request is composed of two parts: a write to an
external origin (sending the request), and then a read from the external
origin (reading the response, which may be restricted under the
same-origin policy).
%
\sys{} adds two new checks: before initiating a
request, we check to see if the label of the origin subsumes the label
of compartment, and if the reader attempts to read the request, we check
to see if the label of the compartment subsumes the label of the response
(which may be explicitly labeled with the {\tt Document-label} header).

If the server is IFC enabled, it may be interested in the current label
of the compartment which initiated the request, to determine the
sensitivity\footnote{And integrity. In fact, mostly integrity.} of the request.
%
Thus, the XHR request is also labeled with a {\tt Document-label} header.
%
(These headers are also applied to loaded content and form submissions.)
%
We extended the send function of XHR to also accept labeled blobs,
which explicitly specify the transmitted label header (and disallow
JavaScript from manually setting this header).

Since normal XHR always checks the same-origin policy, it would be
useful to provide \emph{labeled XHR} which relaxes this requirement.\footnote{As a small note, we still enforce other browser security policies such as CSP.}
%
Intuitively, a safe way to relax the same-origin policy is to allow
cross-origin code to read the result of a response, as long as it
commits to performing no further communication.
%
This must include communication to the \emph{same} origin, since these requests
carry the risk of self-exfiltration~\cite{selfex} and cross-site request
forgery~\cite{CSRF}.
%
To fulfill this requirement, labeled XHR returns its result as a labeled
blob, protected with a server-specified label that defaults to a \emph{fresh}
origin (Section~\ref{sec:privileges}).
%
A server that is protected against self-exfiltration and cross-site
request forgery using end-to-end labels can relax this requirement by
only labeling a result with its own origin.

In an ideal confinement system, it would always be safe to let untrusted
code compute on sensitive data.
%
Unfortunately, practical systems typically have covert channels which
may be exploited to leak sensitive data.
%
Hence, keeping with HiStar~\cite{Zeldovich:2006},
Hails~\cite{giffin:2012:hails}, and Breeze~\cite{Breeze13}, we associate
a \emph{clearance} with every compartment: the compartment is not
allowed to raise its label higher than the clearance or raise its
clearance without an appropriate privilege.
%
The default clearance is \emph{user-controlled}; a user can thus disable
labeled XHR by configuring the clearance of a page to be just
its origin.
%
A user can also selectively enable labeled XHR on specific sites; a
mash-up would ask the user to raise its clearance accordingly.
%
Importantly, raising the clearance does \emph{not} give a mash-up free reign:
the code is still confined.
%
It can only exfiltrate data via a covert channel while the user
is on the web site---a strict improvement over existing mash-ups which
\emph{ask for your password}.

% Local Variables:
% TeX-master: "main.ltx"
% TeX-command-default: "Make"
% End:

