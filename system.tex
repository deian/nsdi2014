\section{The \sys{} system}
\label{sec:system}

We describe the \sys{} design through three concrete examples: a
password-strength checker (section~\ref{sec:system:worker}), a
password manager (section~\ref{sec:system:iframe}), and an extension
that converts phone numbers to links
(section~\ref{sec:system:script}).
%
For completeness, we summarize the different system components in
Table~\toref{table:components}.

\subsection{Isolating third-party code}
\label{sec:system:worker}
%Password-strength checker: 

Password-strength checkers are common to many websites;
%
given a password, a checker computes the strength of the input,
according to some metric.
%
Naturally, when using third-party password-strength checkers
(e.g.,~\tocite{checker1, checker2}) we would like to ensure that the
code does not leak the password.
%
A malicious checker should at worst lie about the strength of a
password.

With server-side support and existing browser mechanisms, we can
incorporate and partially confine a simple third-party strength
checker.
%
Specifically, we must host the checker source on a trusted domain and
execute the code in a Worker; since Workers cannot access the DOM,
this ensures that the password cannot be leaked by e.g., inserting
images, loading scripts, etc.
%
As Workers have access to the \xhr{} constructor, the checker Worker
must, in turn, be created in an iframe that has a CSP policy
restricting network communication (e.g., \texttt{connect-src 'self'}
if the checker source is hosted on the same domain as the page
incorporating it); this is necessary since Workers inherit the CSP
policy of the owner document~\tocite{csp}.
%
Finally, we need to implement the messaging layers between the Worker,
iframe incorporating the checker, and main page.

This approach suffers from four main reasons.
%
First, to avoid trusting the checker website the source code must be
hosted by the library user, wherein a CSP policy can be used to
restrict the checker's exfiltration capabilities. 
%
This need for server-side support is an issue when developers do not
have access on the server to set such headers or host arbitrary code.
%
Second, the fact that we require server-side support to set policies
that restrict communication means that the password-checker cannot
incorporate code it itself does not trust---the same-origin policy
does not allow for such scenarios.
%
Third, the CSP policy of the checker cannot be more restricting than
that of the owner document---hence, the checker can potentially leak
carry out a self-exfiltration attack~\tocite{self-exifiltration} and
leak the password to a public page on the trusted domain.
%
Finally, this approach cannot be used to confine a password
strength-checker that, for example, fetches a list of commonly used
passwords before checking the strength of the password---a completely
safe operation.



\subsection{Confining iframes}
\label{sec:system:iframe}
%Password manager 

\subsection{Privilege separation within iframes}
\label{sec:system:script}
%Phone2Links extension
